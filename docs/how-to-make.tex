\newcommand{\NWtarget}[2]{\hypertarget{#1}{#2}}
\newcommand{\NWlink}[2]{\hyperlink{#1}{#2}}
\newcommand{\NWtxtMacroDefBy}{Fragment defined by}
\newcommand{\NWtxtMacroRefIn}{Fragment referenced in}
\newcommand{\NWtxtMacroNoRef}{Fragment never referenced}
\newcommand{\NWtxtDefBy}{Defined by}
\newcommand{\NWtxtRefIn}{Referenced in}
\newcommand{\NWtxtNoRef}{Not referenced}
\newcommand{\NWtxtFileDefBy}{File defined by}
\newcommand{\NWtxtIdentDefinedIn}{defined in}
\newcommand{\NWtxtIdentUsedIn}{used in}
\newcommand{\NWtxtIdentUsers}{Users:}
\newcommand{\NWtxtIdentsNotUsed}{never used}
\newcommand{\NWtxtIdentsUsed}{Uses:}
\newcommand{\NWsep}{${\diamond}$}
\newcommand{\NWnotglobal}{(not defined globally)}
\newcommand{\NWuseHyperlinks}{}
\documentclass[a4paper, 12pt]{article}
\usepackage{fullpage} % for 1.5 cm margins
\renewcommand{\familydefault}{\sfdefault} % so it doesn't look like LaTeX
\usepackage{helvet}
\usepackage{graphicx}
\graphicspath{ {imgs/} }
\usepackage{float} % so the figures stay with the text

\usepackage{abstract}
\renewcommand{\abstractname}{Overview}
\raggedright

\usepackage{parskip}

\usepackage{hyperref}
\hypersetup{
    colorlinks=true,
    linkcolor=blue,
    filecolor=magenta,      
    urlcolor=cyan,
}

\usepackage{listings}
\usepackage{color}
\lstset{language=C++,
        basicstyle=\ttfamily,
        keywordstyle=\color{blue}\ttfamily,
        stringstyle=\color{red}\ttfamily,
        commentstyle=\color{green}\ttfamily,
        morecomment=[l][\color{magenta}]{\#}
}

\title{Promethean Temperature Sensor}
\author{Joe Collins}

\begin{document}
\maketitle
%%%%%%%%%%%%%%%%%%%%%%%%%%%%%%%%%%%%%%%%%%%%%%%%%%%%%%%%%%%%
\tableofcontents
\clearpage

%%%%%%%%%%%%%%%%%%%%%%%%%%%%%%%%%%%%%%%%%%%%%%%%%%%%%%%%%%%%
\section{Objective}

A temperature sensor that can be scraped/polled using \href{https://prometheus.io/}{Prometheus}.

Prometheus is excellent for monitoring server metrics,
so it makes sense to use it to monitor other metrics such as temperature and humidity.
There doesn't appear to be anything available off the shelf.
There are temperature monitors but some effort would be required to get 
them to integrate with Prometheus.
So we might as well build a bespoke temperature sensor
that can be scraped/polled directly by Prometheus.

\begin{figure}[H]
  \centering
  \includegraphics[width=0.8\textwidth]{prometheus.png}
  \caption{Prometheus scraping sensors}
\end{figure}

With Prometheus scraping the temperature sensors,
a web browser can be used to few graphs of the scraped data.

%%%%%%%%%%%%%%%%%%%%%%%%%%%%%%%%%%%%%%%%%%%%%%%%%%%%%%%%%%%%
\section{Components}

Each sensor is an Arduino microcontroller with a temperature sensore attached,
total cost around \pounds 40.

\begin{tabular}{ll}
  \textbf{Component} & \textbf{Cost} \\ 
  \hline
  Arduino Uno Rev3, ATmega328P, CH340G Compatible Board & \pounds 5.79 \\
  UK 9V AC/DC Power Supply Adapter Plug for Arduino Uno & \pounds 7.95 \\
  Ethernet Shield LAN W5100 for Arduino Uno & \pounds 7.75 \\
  DHT22 AM2302 Digital Temperature and Humidity Sensor & \pounds 6.90 \\
  0.25 Watt Metal Film Resistor 10K Ohm & \pounds 0.99 \\
  Uno Ethernet Shield Case & \pounds 10.36 \\
  \hline
  Total cost in November 2020 & \textbf{\pounds 39.74}  \\
\end{tabular}

\begin{figure}[H]
  \centering
  \includegraphics[width=0.8\textwidth]{components.jpg}
  \caption{Components}
\end{figure}

%%%%%%%%%%%%%%%%%%%%%%%%%%%%%%%%%%%%%%%%%%%%%%%%%%%%%%%%%%%%
\section{Wiring the Sensor}
\label{sec:wiring}

The sensor is a DHT22 AM2302 capacitive humidity sensing digital temperature and humidity module.
It is has a calibrated digital signal output of the temperature and humidity sensors.
Other sensors are available such as the cheaper DHT11,
but supposedly it is less sensitive and less durable.
Whilst sensitivity is not important
durability probably is.

The sensor is wired with a pull up resistor.
This ensures that the signal wire has a small but consistent current,
so it is less susceptible to electrical interference.
In theory you can also set the pin mode with \verb|pinMode("pin", INPUT_PULLUP);|
to use a built in pull up resistor.

The sensor works without the pull up resistor but is probably less accurate
(though I haven't tested it).

\begin{figure}[H]
  \centering
  \includegraphics[width=0.8\textwidth]{wiring-dht22.jpg}
  \caption{Wiring diagram with pull up resistor}
\end{figure}

%%%%%%%%%%%%%%%%%%%%%%%%%%%%%%%%%%%%%%%%%%%%%%%%%%%%%%%%%%%%
\section{Programming}

Tooling: 

\begin{itemize}
  \item VSCode \\
  \url{https://code.visualstudio.com/}
  \item Platformio \\
  \url{https://marketplace.visualstudio.com/items?itemName=platformio.platformio-ide}
\end{itemize}

All Arduino programs have the same format with a \verb|void setup()| and \verb|void loop()| functions.
The \verb|loop()| runs continuously and the \verb|setup()| is run
when the Arduino is turned on
or
when the reset button (red button near the USB socket) is press.

\begin{flushleft} \small
\begin{minipage}{\linewidth}\label{scrap1}\raggedright\small
\NWtarget{nuweb5a}{}\verb@"../src/shit.cpp"@\nobreak\ {\footnotesize{5a}}$\equiv$
\vspace{-1ex}
\begin{list}{}{\setlength{\leftmargin}{1em}} \item
\mbox{}\lstinline@#include <Arduino.h>@\\
\mbox{}\lstinline@@$\langle\,${\itshape libraries}\ {\footnotesize \NWlink{nuweb5b}{5b}},\ ...\,$\rangle\,$\verb@@\\
\mbox{}\lstinline@@$\langle\,${\itshape configuration}\ {\footnotesize \NWlink{nuweb6a}{6a}},\ ...\,$\rangle\,$\verb@@\\
\mbox{}\lstinline@@$\langle\,${\itshape functions}\ {\footnotesize \NWlink{nuweb6c}{6c}},\ ...\,$\rangle\,$\verb@@\\
\mbox{}\lstinline@void setup() @\\
\mbox{}\lstinline@{@\\
\mbox{}\lstinline@  Serial.begin(9600);@\\
\mbox{}\lstinline@  while (!Serial){;}@\\
\mbox{}\lstinline@  @$\langle\,${\itshape setup}\ {\footnotesize \NWlink{nuweb6b}{6b}}\,$\rangle\,$\verb@@\\
\mbox{}\lstinline@}@\\
\mbox{}\lstinline@void loop() @\\
\mbox{}\lstinline@{@\\
\mbox{}\lstinline@  @$\langle\,${\itshape loop}\ {\footnotesize \NWlink{nuweb8a}{8a}}\,$\rangle\,$\verb@@\\
\mbox{}\lstinline@}@\\
\mbox{}{\NWsep}
\end{list}
\vspace{-1ex}
\end{minipage}
\end{flushleft}

Since we're using Platformio we need the Arduino header (\verb|<Arduino.h>|).
For normal Arduino programs it is not required.

It's useful to be able to connect to the Arduino over the USB cable,
so the \verb|setup()| opens 

Include serial for output to monitor serial communications and wait for the port to open.
Waiting for the serial port to connect is only needed for native USB ports.

%%%%%%%%%%%%%%%%%%%%%%%%%%%%%%%%%%%%%%%%%%%%%%%%%%%%%%%%%%%%
\subsection{Sensor}

To interact with the DHT22 sensor we need specific drivers (\verb|<DHT.h>|),
which rely on the Adafruit Unified Sensor Driver (\verb|<Adafruit_Sensor.h>|),
The Adafruit Unified Sensor Driver is an abstraction layer
which makes creating reliable drivers is easier. 

The Serial Peripheral Interface (\verb|<SPI.h>|) is a synchronous serial data protocol
used by microcontrollers for communicating with peripheral devices quickly over short distances.

\begin{flushleft} \small
\begin{minipage}{\linewidth}\label{scrap2}\raggedright\small
\NWtarget{nuweb5b}{}$\langle\,${\itshape libraries}\nobreak\ {\footnotesize{5b}}$\,\rangle\equiv$
\vspace{-1ex}
\begin{list}{}{\setlength{\leftmargin}{1em}} \item
\mbox{}\lstinline@// Sensor libraries:@\\
\mbox{}\lstinline@#include <Adafruit_Sensor.h>@\\
\mbox{}\lstinline@#include <DHT.h>@\\
\mbox{}\lstinline@#include <SPI.h>@\\
\mbox{}{\NWsep}
\end{list}
\vspace{-1ex}
\vspace{-1ex}
\footnotesize
\begin{list}{}{\setlength{\itemsep}{-\parsep}\setlength{\itemindent}{-\leftmargin}}
\item \NWtxtMacroDefBy\ \NWlink{nuweb5b}{5b}, \NWlink{nuweb7b}{7b}.
\item \NWtxtMacroRefIn\ \NWlink{nuweb5a}{5a}.
\end{list}
\end{minipage}
\end{flushleft}

The sensor communicates using pin 2 (see \hyperref[sec:wiring]{wiring})
and
the sensor type need to be initialized.
The temperature and humidity
are stored in global variables
since they'll be used by pretty much every part of the program.

\begin{flushleft} \small
\begin{minipage}{\linewidth}\label{scrap3}\raggedright\small
\NWtarget{nuweb6a}{}$\langle\,${\itshape configuration}\nobreak\ {\footnotesize{6a}}$\,\rangle\equiv$
\vspace{-1ex}
\begin{list}{}{\setlength{\leftmargin}{1em}} \item
\mbox{}\lstinline@#define DHTPIN 2@\\
\mbox{}\lstinline@#define DHTTYPE DHT22@\\
\mbox{}\lstinline@DHT dht = DHT(DHTPIN, DHTTYPE);@\\
\mbox{}\lstinline@float temperature = 0;@\\
\mbox{}\lstinline@float humidity = 0;@\\
\mbox{}{\NWsep}
\end{list}
\vspace{-1ex}
\vspace{-1ex}
\footnotesize
\begin{list}{}{\setlength{\itemsep}{-\parsep}\setlength{\itemindent}{-\leftmargin}}
\item \NWtxtMacroDefBy\ \NWlink{nuweb6a}{6a}, \NWlink{nuweb7c}{7c}.
\item \NWtxtMacroRefIn\ \NWlink{nuweb5a}{5a}.
\end{list}
\end{minipage}
\end{flushleft}

The sensor needs to \verb|begin()|.
Formerly this method as used to pass in parameters relating the the speed of the Arduino.
Now the sensor automatically adapts
but \verb|begin()| is still needed.

\begin{flushleft} \small
\begin{minipage}{\linewidth}\label{scrap4}\raggedright\small
\NWtarget{nuweb6b}{}$\langle\,${\itshape setup}\nobreak\ {\footnotesize{6b}}$\,\rangle\equiv$
\vspace{-1ex}
\begin{list}{}{\setlength{\leftmargin}{1em}} \item
\mbox{}\lstinline@  Serial.println("Set up sensor");@\\
\mbox{}\lstinline@  dht.begin();@\\
\mbox{}{\NWsep}
\end{list}
\vspace{-1ex}
\vspace{-1ex}
\footnotesize
\begin{list}{}{\setlength{\itemsep}{-\parsep}\setlength{\itemindent}{-\leftmargin}}
\item \NWtxtMacroRefIn\ \NWlink{nuweb5a}{5a}.
\end{list}
\end{minipage}
\end{flushleft}

Set the values in the globals,
could have passed stuff around.
Read the humidity as a percentage
Read the temperature as Celsius:

\begin{flushleft} \small
\begin{minipage}{\linewidth}\label{scrap5}\raggedright\small
\NWtarget{nuweb6c}{}$\langle\,${\itshape functions}\nobreak\ {\footnotesize{6c}}$\,\rangle\equiv$
\vspace{-1ex}
\begin{list}{}{\setlength{\leftmargin}{1em}} \item
\mbox{}\lstinline@void readSensor()@\\
\mbox{}\lstinline@{@\\
\mbox{}\lstinline@@\\
\mbox{}\lstinline@  humidity = dht.readHumidity();@\\
\mbox{}\lstinline@@\\
\mbox{}\lstinline@  temperature = dht.readTemperature();@\\
\mbox{}\lstinline@  // Check if any reads failed and exit early (to try again):@\\
\mbox{}\lstinline@  if (isnan(humidity) || isnan(temperature)) @\\
\mbox{}\lstinline@  {@\\
\mbox{}\lstinline@    Serial.println("Failed to read from sensor");@\\
\mbox{}\lstinline@    return;@\\
\mbox{}\lstinline@  }@\\
\mbox{}\lstinline@}@\\
\mbox{}{\NWsep}
\end{list}
\vspace{-1ex}
\vspace{-1ex}
\footnotesize
\begin{list}{}{\setlength{\itemsep}{-\parsep}\setlength{\itemindent}{-\leftmargin}}
\item \NWtxtMacroDefBy\ \NWlink{nuweb6c}{6c}, \NWlink{nuweb7a}{7a}, \NWlink{nuweb8b}{8b}.
\item \NWtxtMacroRefIn\ \NWlink{nuweb5a}{5a}.
\end{list}
\end{minipage}
\end{flushleft}

Not necessary in use but handy for development.

\begin{flushleft} \small
\begin{minipage}{\linewidth}\label{scrap6}\raggedright\small
\NWtarget{nuweb7a}{}$\langle\,${\itshape functions}\nobreak\ {\footnotesize{7a}}$\,\rangle\equiv$
\vspace{-1ex}
\begin{list}{}{\setlength{\leftmargin}{1em}} \item
\mbox{}\lstinline@void serialPrintReadings()@\\
\mbox{}\lstinline@{@\\
\mbox{}\lstinline@  Serial.print("Humidity: ");@\\
\mbox{}\lstinline@  Serial.print(humidity);@\\
\mbox{}\lstinline@  Serial.print(" % | ");@\\
\mbox{}\lstinline@  Serial.print("Temperature: ");@\\
\mbox{}\lstinline@  Serial.print(temperature);@\\
\mbox{}\lstinline@  Serial.println(" C");@\\
\mbox{}\lstinline@}@\\
\mbox{}{\NWsep}
\end{list}
\vspace{-1ex}
\vspace{-1ex}
\footnotesize
\begin{list}{}{\setlength{\itemsep}{-\parsep}\setlength{\itemindent}{-\leftmargin}}
\item \NWtxtMacroDefBy\ \NWlink{nuweb6c}{6c}, \NWlink{nuweb7a}{7a}, \NWlink{nuweb8b}{8b}.
\item \NWtxtMacroRefIn\ \NWlink{nuweb5a}{5a}.
\end{list}
\end{minipage}
\end{flushleft}

%%%%%%%%%%%%%%%%%%%%%%%%%%%%%%%%%%%%%%%%%%%%%%%%%%%%%%%%%%%%
\subsection{Ethernet Client}

aWOT is in version 3

\verb|lasselukkari/aWOT@0.0.0-alpha+sha.bf07e6371c|


\begin{flushleft} \small
\begin{minipage}{\linewidth}\label{scrap7}\raggedright\small
\NWtarget{nuweb7b}{}$\langle\,${\itshape libraries}\nobreak\ {\footnotesize{7b}}$\,\rangle\equiv$
\vspace{-1ex}
\begin{list}{}{\setlength{\leftmargin}{1em}} \item
\mbox{}\lstinline@// Ethernet shield@\\
\mbox{}\lstinline@#include <Ethernet.h>@\\
\mbox{}\lstinline@#include <aWOT.h>@\\
\mbox{}{\NWsep}
\end{list}
\vspace{-1ex}
\vspace{-1ex}
\footnotesize
\begin{list}{}{\setlength{\itemsep}{-\parsep}\setlength{\itemindent}{-\leftmargin}}
\item \NWtxtMacroDefBy\ \NWlink{nuweb5b}{5b}, \NWlink{nuweb7b}{7b}.
\item \NWtxtMacroRefIn\ \NWlink{nuweb5a}{5a}.
\end{list}
\end{minipage}
\end{flushleft}


\begin{flushleft} \small
\begin{minipage}{\linewidth}\label{scrap8}\raggedright\small
\NWtarget{nuweb7c}{}$\langle\,${\itshape configuration}\nobreak\ {\footnotesize{7c}}$\,\rangle\equiv$
\vspace{-1ex}
\begin{list}{}{\setlength{\leftmargin}{1em}} \item
\mbox{}\lstinline@// Enter a MAC address and IP address for your controller below.@\\
\mbox{}\lstinline@// The IP address will be dependent on your local network:@\\
\mbox{}\lstinline@byte mac[] = {@\\
\mbox{}\lstinline@  0xDE, 0xAD, 0xBE, 0xEF, 0xFE, 0xED@\\
\mbox{}\lstinline@};@\\
\mbox{}\lstinline@IPAddress ip(10, 0, 21, 211);@\\
\mbox{}\lstinline@// Initialize the Ethernet server library@\\
\mbox{}\lstinline@// with the IP address and port you want to use@\\
\mbox{}\lstinline@// (port 80 is default for HTTP):@\\
\mbox{}\lstinline@EthernetServer server(80);@\\
\mbox{}\lstinline@Application app;@\\
\mbox{}{\NWsep}
\end{list}
\vspace{-1ex}
\vspace{-1ex}
\footnotesize
\begin{list}{}{\setlength{\itemsep}{-\parsep}\setlength{\itemindent}{-\leftmargin}}
\item \NWtxtMacroDefBy\ \NWlink{nuweb6a}{6a}, \NWlink{nuweb7c}{7c}.
\item \NWtxtMacroRefIn\ \NWlink{nuweb5a}{5a}.
\end{list}
\end{minipage}
\end{flushleft}

No need to over do the rate besides it takes abot about 250 milliseconds to poll the sensor.
Prometheus normally scrapes every minute.

\begin{flushleft} \small
\begin{minipage}{\linewidth}\label{scrap9}\raggedright\small
\NWtarget{nuweb8a}{}$\langle\,${\itshape loop}\nobreak\ {\footnotesize{8a}}$\,\rangle\equiv$
\vspace{-1ex}
\begin{list}{}{\setlength{\leftmargin}{1em}} \item
\mbox{}\lstinline@  // Wait a couple of seconds between measurements.@\\
\mbox{}\lstinline@  delay(2000);@\\
\mbox{}\lstinline@  // Reading temperature or humidity takes about 250 milliseconds@\\
\mbox{}\lstinline@  // Sensor readings may also be up to 2 seconds 'old'@\\
\mbox{}\lstinline@  readSensor();@\\
\mbox{}\lstinline@  serialPrintReadings();@\\
\mbox{}\lstinline@  EthernetClient client = server.available();@\\
\mbox{}\lstinline@  if (client.connected()) {@\\
\mbox{}\lstinline@    app.process(&client);@\\
\mbox{}\lstinline@    client.stop();@\\
\mbox{}\lstinline@  }@\\
\mbox{}{\NWsep}
\end{list}
\vspace{-1ex}
\vspace{-1ex}
\footnotesize
\begin{list}{}{\setlength{\itemsep}{-\parsep}\setlength{\itemindent}{-\leftmargin}}
\item \NWtxtMacroRefIn\ \NWlink{nuweb5a}{5a}.
\end{list}
\end{minipage}
\end{flushleft}

Prometheus style metrics for scraping.

\begin{flushleft} \small
\begin{minipage}{\linewidth}\label{scrap10}\raggedright\small
\NWtarget{nuweb8b}{}$\langle\,${\itshape functions}\nobreak\ {\footnotesize{8b}}$\,\rangle\equiv$
\vspace{-1ex}
\begin{list}{}{\setlength{\leftmargin}{1em}} \item
\mbox{}\lstinline@void metricsCmd(Request &req, Response &res)@\\
\mbox{}\lstinline@{@\\
\mbox{}\lstinline@  Serial.println("Request for metrics");@\\
\mbox{}\lstinline@  res.set("Content-Type", "text/plain");@\\
\mbox{}\lstinline@  res.print("# HELP temperature is the last temperature reading in degrees celsius\n");@\\
\mbox{}\lstinline@  res.print("# TYPE temp gauge\n");@\\
\mbox{}\lstinline@  res.print("temperature " + String(temperature) + "\n");@\\
\mbox{}\lstinline@  res.print("# HELP humidity is the last relative humidity reading as a percentage\n");@\\
\mbox{}\lstinline@  res.print("# TYPE humidity gauge\n");@\\
\mbox{}\lstinline@  res.print("humidity " + String(humidity) + "\n");@\\
\mbox{}\lstinline@}@\\
\mbox{}{\NWsep}
\end{list}
\vspace{-1ex}
\vspace{-1ex}
\footnotesize
\begin{list}{}{\setlength{\itemsep}{-\parsep}\setlength{\itemindent}{-\leftmargin}}
\item \NWtxtMacroDefBy\ \NWlink{nuweb6c}{6c}, \NWlink{nuweb7a}{7a}, \NWlink{nuweb8b}{8b}.
\item \NWtxtMacroRefIn\ \NWlink{nuweb5a}{5a}.
\end{list}
\end{minipage}
\end{flushleft}

%%%%%%%%%%%%%%%%%%%%%%%%%%%%%%%%%%%%%%%%%%%%%%%%%%%%%%%%%%%%
\subsection{Upload}

Drivers on PC.

read back doesn't always work.

%%%%%%%%%%%%%%%%%%%%%%%%%%%%%%%%%%%%%%%%%%%%%%%%%%%%%%%%%%%%
\subsection{Testing}

\begin{verbatim}
  --- Quit: Ctrl+C | Menu: Ctrl+T | Help: Ctrl+T followed by Ctrl+H ---
  Webserver set up
  Sensor set up
  Humidity: 55.90 % | Temperature: 22.10 C
  Humidity: 56.30 % | Temperature: 22.20 C
\end{verbatim}


Rather than link up to a router

\begin{itemize}
  \item Assign a manual IP address to the laptop's ethernet connection say 10.0.21.1.
  \item Subnet mask 255.255.255.0.
  \item Assign a manual IP address to the Arduino's ethernet, say 10.0.21.211.
  \item Subnet mask 255.255.255.0.
  \item Leave the default Gateway empty.
  \item Use an ethernet patch cable to link the two (since 100BaseT onwards it doesn't have to be a special cross over cable).
  \item You should then be able to get your Arduino site up on \url{http://192.168.0.2} from the laptop.
\end{itemize}
  
This is the endpoint at \url{http://10.0.21.211/metrics}.
  
\begin{verbatim}
  > curl 10.0.21.211
  # HELP temperature is the last temperature reading in degrees celsius
  # TYPE temp gauge
  temperature 23.30
  # HELP humidity is the last relative humidity reading as a percentage
  # TYPE humidity gauge
  humidity 47.60
\end{verbatim}

\verb|docker-compose up|

\begin{figure}[H]
  \centering
  \includegraphics[width=0.8\textwidth]{graph.jpg}
  \caption{Temperature graph}
\end{figure}
  

%%%%%%%%%%%%%%%%%%%%%%%%%%%%%%%%%%%%%%%%%%%%%%%%%%%%%%%%%%%%
\section{Packaging}

\begin{figure}[H]
  \centering
  \includegraphics[width=0.8\textwidth]{sensor-mount.jpg}
  \caption{Components for mounting the sensor}
\end{figure}

2.5 mm holes and 4 mm holes.

\begin{figure}[H]
  \centering
  \includegraphics[width=0.8\textwidth]{packaging.jpg}
  \caption{Mounted sensor and wiring}
\end{figure}

\end{document}